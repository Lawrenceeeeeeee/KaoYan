\documentclass[UTF8]{ctexart}
\usepackage{amsmath}
\usepackage{amssymb}
\usepackage{geometry}

\geometry{a4paper, left=2.5cm, right=2.5cm, top=2.5cm, bottom=2.5cm}

% --- 禁止公式内换行 ---
\relpenalty=10000
\binoppenalty=10000
% --------------------

\begin{document}

\title{2025年全国硕士研究生招生考试试题 \\ (数学二)}
\author{科目代码: 302}
\date{}
\maketitle

\section*{一、选择题: $1\sim10$ 小题, 每小题5分, 共50分。}
\begin{enumerate}

\item 设函数 $z=z(x,y)$ 由 $z+\ln z-\int_{y}^{x}e^{-t^{2}}dt=0$ 确定, 则 $\frac{\partial z}{\partial x}+\frac{\partial z}{\partial y}=$ ( \quad )

    A. $\frac{z}{z+1}(e^{-x^{2}}-e^{-y^{2}})$ \qquad B. $\frac{z}{z+1}(e^{-x^{2}}+e^{-y^{2}})$

    C. $-\frac{z}{z+1}(e^{-x^{2}}-e^{-y^{2}})$ \qquad D. $-\frac{z}{z+1}(e^{-x^{2}}+e^{-y^{2}})$

\item 已知函数 $f(x)=\int_{0}^{x}{e^{t}}^{2}\sin t dt$, $g(x) = \int_{0}^{x}e^{t^{2}}dt-\sin^{2}x$, 则( \quad )

    A. $x=0$ 是 $f(x)$ 的极值点, 也是 $g(x)$ 的极值点

    B. $x=0$ 是 $f(x)$ 的极值点, $(0,0)$ 是曲线 $y=g(x)$ 的拐点

    C. $x=0$ 是 $f(x)$ 的极值点, $(0,0)$ 是曲线 $y=f(x)$ 的拐点

    D. $(0,0)$ 是曲线 $y=f(x)$ 的拐点, 也是曲线 $y=g(x)$ 的拐点

\item 如果对微分方程 $y''-2ay'+(a+2)y=0$ 的任一解 $y(x)$, 反常积分 $\int_{0}^{+\infty}y(x)dx$ 均收敛, 那么a的取值范围是( \quad )

    A. $(-2,-1]$ \qquad B. $(-\infty,-1]$ \qquad C. $(-2,0)$ \qquad D. $(-\infty,0)$

\item 设函数 $f(x), g(x)$ 在 $x=0$ 的某去心邻域内有定义且恒不为零, 若当 $x\rightarrow0$ 时, $f(x)$ 是 $g(x)$ 的高阶无穷小, 则当 $x\rightarrow0$ 时,( \quad )

    A. $f(x)+g(x)=o(g(x))$ \qquad B. $f(x)g(x)=o(f^{2}(x))$

    C. $f(x)=o(e^{g(x)}-1)$ \qquad D. $f(x)=o(g^{2}(x))$

\item 设函数 $f(x,y)$ 连续, 则 $\int_{-2}^{2}dx\int_{4-x^{2}}^{4}f(x,y)dy=$ ( \quad )

    A. $\int_{0}^{4}[\int_{-2}^{-\sqrt{4-y}}f(x,y)dx+\int_{\sqrt{4-y}}^{2}f(x,y)dx]dy$

    B. $\int_{0}^{4}[\int_{-2}^{\sqrt{4-y}}f(x,y)dx+\int_{\sqrt{4-y}}^{2}f(x,y)dx]dy$

    C. $\int_{0}^{4}[\int_{-2}^{-\sqrt{4-y}}f(x,y)dx+\int_{2}^{\sqrt{4-y}}f(x,y)dx]dy$

    D. $2\int_{0}^{4}dy\int_{\sqrt{4-y}}^{2}f(x,y)dx$

\item 设单位质点 $P,Q$ 分别位于点(0,0)和(0,1)处, P从点(0,0)出发沿x轴正向移动, 记G为引力常量, 则当质点P移动到点(1.0)时, 克服质点Q的引力所做的功为( \quad )

    A. $\int_{0}^{1}\frac{G}{x^{2}+1}dx$ \qquad B. $\int_{0}^{1}\frac{Gx}{(x^{2}+1)^{\frac{3}{2}}}dx$
    
    C. $\int_{0}^{1}\frac{G}{(x^{2}+1)^{\frac{3}{2}}}dx$ \qquad D. $\int_{0}^{1}\frac{G(x+1)}{(x^{2}+1)^{\frac{3}{2}}}dx$

\item 设函数 $f(x)$ 连续, 给出下列四个条件, 其中能得到“$f(x)$ 在 $x=0$ 处可导”的条件个数是( \quad ) \\
    ① $\lim_{x\rightarrow0}\frac{|f(x)|-f(0)}{x}$ 存在; \quad ② $\lim_{x\rightarrow0}\frac{f(x)-|f(0)|}{x}$ 存在; \\
    ③ $\lim_{x\rightarrow0}\frac{|f(x)|}{x}$ 存在; \quad ④ $\lim_{x\rightarrow0}\frac{|f(x)|-|f(0)|}{x}$ 存在;

    A. 1 \qquad B. 2 \qquad C. 3 \qquad D. 4

\item 设矩阵 $\begin{pmatrix}1&2&0\\ 2&a&0\\ 0&0&b\end{pmatrix}$ 有一个正特征值和两个负特征值, 则( \quad )

    A. $a>4,b>0$ \qquad B. $a<4,b>0$ \qquad C. $a>4,b<0$ \qquad D. $a<4,b<0$

\item 下列矩阵中, 可以经过若干初等行变换得到矩阵 $\begin{pmatrix}1&1&0&1\\ 0&0&1&2\\ 0&0&0&0\end{pmatrix}$ 的是( \quad )

    A. $\begin{pmatrix}1&1&0&1\\ 1&2&1&3\\ 2&3&1&4\end{pmatrix}$ \quad B. $\begin{pmatrix}1&1&0&1\\ 1&1&2&5\\ 1&1&1&3\end{pmatrix}$ \quad C. $\begin{pmatrix}1&0&0&1\\ 0&1&0&3\\ 0&1&0&0\end{pmatrix}$ \quad D. $\begin{pmatrix}1&1&2&3\\ 1&2&2&3\\ 2&3&4&6\end{pmatrix}$

\item 设3阶矩阵A, B满足 $r(AB)=r(BA)+1$, 则( \quad )

    A. 方程组 $(A+B)x=0$ 只有零解

    B. 方程组 $Ax=0$ 与方程组 $Bx=0$ 均只有零解

    C. 方程组 $Ax=0$ 与方程组 $Bx=0$ 没有公共非零解

    D. 方程组 $ABAx=0$ 与方程组 $BABx=0$ 有公共非零解

\end{enumerate}

\section*{二、填空题: $11\sim16$ 小题, 每小题5分, 共30分。}
\begin{enumerate}
    \setcounter{enumi}{10}

    \item 设 $\int_{1}^{+\infty}\frac{a}{x(2x+a)}dx=\ln 2$. 则 a = \underline{\hspace{3cm}}.
    
    \item 曲线 $y=\sqrt[3]{x^{3}-3x^{2}+1}$ 的渐近线方程为 \underline{\hspace{3cm}}.

    \item $\lim_{n\rightarrow\infty}\frac{1}{n^{2}}[\ln\frac{1}{n}+2\ln\frac{2}{n}+\cdot\cdot\cdot+(n-1)\ln\frac{n-1}{n}]=$ \underline{\hspace{3cm}}.

    \item 已知函数 $y=y(x)$ 由 $\begin{cases}x=\ln(1+2t)\\ 2t-\int_{t}^{y+t^{2}}e^{-u^{2}}du=0\end{cases}$ 确定, 则 $\frac{dy}{dx}|_{t=0}=$ \underline{\hspace{3cm}}.

    \item 微分方程 $(2y-3x)dx+(2x-5y)dy=0$ 满足条件 $y(1)=1$ 的解为 \underline{\hspace{3cm}}.

    \item 设矩阵 $A=(\alpha_{1},\alpha_{2},\alpha_{3},\alpha_{4})$, 若 $\alpha_{1},\alpha_{2},\alpha_{3}$ 线性无关, 且 $\alpha_{1}+\alpha_{2}=\alpha_{3}+\alpha_{4}$, 则方程组 $Ax=\alpha_{1}+4\alpha_{4}$ 的通解为 $x=$ \underline{\hspace{3cm}}.
\end{enumerate}

\section*{三、解答题: $17\sim22$小题, 共70分。}
\begin{description}

    \item[17. (本题满分10分)] ~\\
    计算 $\int_{0}^{1}\frac{1}{(x+1)(x^{2}-2x+2)}dx$.

    \item[18. (本题满分12分)] ~\\
    设函数 $f(x)$ 在 $x=0$ 处连续, 且 $\lim_{x\rightarrow0}\frac{xf(x)-e^{2\sin x}+1}{\ln(1+x)+\ln(1-x)}=-3$, 证明 $f(x)$ 在 $x=0$ 处可导, 并求 $f'(0)$.
    
    \item[19. (本题满分12分)] ~\\
    设函数 $f(x,y)$ 可微且满足 $df(x,y)=-2xe^{-y}dx+e^{-y}(x^{2}-y-1)dy$, $f(0,0)=2$. 求 $f(x,y)$ 并求 $f(x,y)$ 的极值.

    \item[20. (本题满分12分)] ~\\
    已知平面有界区域 $D=\{(x,y)|x^{2}+y^{2}\le4x,x^{2}+y^{2}\le4y\}$, 计算 $\iint_{D}(x-y)^{2}dxdy$.
    
    \item[21. (本题满分12分)] ~\\
    设函数 $f(x)$ 在区间 $(a,b)$ 内可导, 证明导函数 $f'(x)$ 在 $(a,b)$ 内严格单调增加的充分必要条件是: 对 $(a,b)$ 内任意的 $x_{1},x_{2},x_{3}$ 当 $x_{1}<x_{2}<x_{3}$ 时, $\frac{f(x_{2})-f(x_{1})}{x_{2}-x_{1}}<\frac{f(x_{3})-f(x_{2})}{x_{3}-x_{2}}$.

    \item[22. (本题满分12分)] ~\\
    已知矩阵 $A=\begin{pmatrix}4&1&-2\\ 1&1&1\\ -2&1&a\end{pmatrix}$ 与 $B=\begin{pmatrix}k&0&0\\ 0&6&0\\ 0&0&0\end{pmatrix}$ 合同. \\
    (1) 求a的值及k的取值范围; \\
    (2) 若存在正交矩阵Q 使得 $Q^{T}AQ=B$, 求k及Q.

\end{description}

\end{document}